\documentclass[10pt,a4paper]{article}
\usepackage[utf8]{inputenc}
\usepackage{amsmath}
\usepackage{amsfonts}
\usepackage{amssymb}
\usepackage{graphicx}
\usepackage{bm}
\usepackage{enumerate}
\usepackage{units}

%TODO alternative for \pmb{\mathbb{E}} (pmb is ugly)
%TODO same with \pmb\varepsilon

\begin{document}

\section{Literature}

see handout

\section{Transient structural problems in 2d/ 3d assuming linear elasticity}

Strong form of the balance of momentum
\begin{equation}
\text{div} ( \bm\sigma (\bm x , t)) + \varrho (\bm x) b (\bm x, t)
= \underbrace{\varrho (x) \bm{\ddot{u}} (\bm x, t)}_{\substack{\text{inertial forces}}}
~~\forall~\bm x  \in B,~t \in [0,T]
\label{eq:strformbalmom}
\end{equation}

%TODO

problem statement: \\
find $\bm u (\bm{x}, t)$ that solves \eqref{eq:strformbalmom} subject to:

\begin{enumerate}[i)]
\item boundary conditions (b.c.)
%TODO add  picture
\begin{equation}
\tag{Dirichlet}
\bm u (\bm{x}, t) = \bm{\bar{u}} (\bm{x}, t) ~\text{on}~ \partial B^u~~\forall~t \in [0, T]
\end{equation}
\begin{equation}
\tag{Neumann}
\bm{t} (\bm{x}, t) = \bm{\sigma} (\bm{x}, t) \bm{n} (\bm{x}) = \bm{\bar{t}} (\bm{x}, t) ~\text{on}~ \partial B^t ~~\forall~t \in [0, T]
\end{equation}

$\partial B = \partial B^u \cup \partial B^t$ \\
$\partial B^u \cap \partial B^t = \emptyset$
%TODO toadd
\item initial conditions (i.c)
\[ \bm{u} (\bm{x}, 0) = \bm{\overline{u_0}} (\bm{x}) ~~\forall \bm{x} \in B \]
\[ \bm{\dot{u}} (\bm{x}, 0) = \bm{\overline{v_0}} (\bm{x}) ~~\forall \bm{x} \in B \]
\item kinematics (small strain theory)
\[ \pmb\varepsilon (\bm{x}, t) = \nabla^{sym} \bm{u} (\bm{x}, t) = \frac{1}{2} \left[ \nabla \bm{u} + [\nabla \bm{u}]^T \right] \]
\item constitute relations (material model)

e.g. Hooke's law: $\bm{\sigma} = \pmb{\mathbb{E}}^e : \pmb\varepsilon$
\end{enumerate}

Derivation for the weak form (standard Galerkin approach)
\begin{itemize}
\item multiplication by test function $\bm{\eta}$
\[ \Rightarrow \bm{\eta} \cdot \operatorname{div}(\bm{\sigma}) + \varrho \bm{\eta} \cdot \bm{b} = \varrho \bm{\eta} \cdot \bm{\ddot{u}} \]
\item integration over the spatial domain B
\[ \Rightarrow \int_B [ \bm{\eta} \cdot \operatorname{div}(\bm\sigma) + \varrho \bm\eta \cdot \bm b ] ~dv
= \underbrace{\int_B \varrho \bm\eta \cdot \bm{\ddot{u}} ~dv}_{\substack{W_{inert}}}\]
for $\bm\eta = \delta \bm u$ (virtual displacement), $W_{inert}$ can be interpreted as the virtual work of the "inertial forces".
\item \textit{spatial} discretization of the domain
\[ B \approx B^h = \bigcup_{e=1}^{M_{el}} B^e \]
\textit{spatial} approximation of test functions
\[ \bm\eta(\bm x) \approx \bm\eta^h(\bm x) = \bigcup_{e=1}^{M_{el}} \bm\eta^e (\bm x) \]
with
\[ \bm\eta^e(\bm x) = \sum_{A=1}^{M_{en}} \bm\eta^{eA} N^A(\bm x) \]
\end{itemize}

Let us consider the term $W_{inter}$ (only difference to the euqations considered last semester) in more detail.

\begin{equation}
\begin{aligned}
W_{inter}^h &= \sum_{l=1}^{M_{el}} \left[ \int_{B^e} \left[ \sum_{A=1}^{M_{en}} N^A \bm\eta^{eA} \cdot \bm{\ddot{u}} \right] dv \right] \\
&= \sum_{e=1}^{M_{el}} \sum_{A=1}^{M_{en}} \bm\eta^{eA} \underbrace{\int_{B^e} N^A \varrho \bm{\ddot{u}} ~dv}_{=: f_{inert,e}^A}
\end{aligned}
\notag
\end{equation}

Connection to \textit{global} list via
\[ \bm\eta^{eA} = \bm\eta^t \pmb{\mathbb{L}}_{e,A}^t \]
\begin{equation}
\begin{aligned}
\Rightarrow W_{inert}^h &= \sum_{e=1}^{M_{el}} \sum_{A=1}^{M_{en}} \bm\eta^t \pmb{\mathbb{L}}_{e,A}^t \cdot \bm f_{inert,e}^A \\
&= \bm\eta^t
\underbrace{\left[ \sum_{e=1}^{M_{el}} \sum_{A=1}^{M_{en}} \pmb{\mathbb{L}}_{e,A}^t \cdot \bm f_{inert,e}^A \right]}_{=: \bm f_{inert}}
\end{aligned}
\notag
\end{equation}

Together with what was de???? for the quasi-static case, we obtain
\[ W^h = \bm\eta^t [ \bm f_{inert} + \bm f_{int} - \bm f_{vol} - \bm f_{sur} ] \overset{!}{=} \bm 0 \]

and thus it follows for arbitrary $\bm\eta^t \not\equiv \bm 0$
\[ \bm f_{inert} + \bm f{int} - \bm f_{vol} - \bm f_{sur} = \bm 0 \]
note: $\bm f_{int} = \bm{\hat{f}}_{int} (\bm u)$ and $\bm f_{inert} = \bm{\hat{f}}_{inert} (\bm{\ddot{u}})$

We already know the representation
\[ \bm f_{int} = \pmb{\mathbb{K}} \bm u \]
with $\pmb{\mathbb{K}} := \sum_{e=1}^{n_{el}} \sum_{A=1}^{M_{en}}$







%TODO very much
%start of second lesson

1. Semi-discrete equations of motion

\[ \bm t_{inert} = \sum_{e=1}^{n_{el}} \sum_{A=1}^{n_{en}} \bm L_{e,A}^t \int_.....
= \sum_{e=1}^{n_{el}} \sum_{A=1}^{n_{en}} \bm L_{e,A}^t \int_{B^e} N^A \varphi \bm L_{e,B} \bm{\ddot{u}} N^B dV
= \sum_{e=1}^{n_{el}} \sum_{A=1}^{n_{en}} \sum_{B=1}^{n_{en}}
\]
%TODO

{Remarks:} \\
\begin{itemize}
\item Damping is often taken into account by an additional term.
\[ \bm M \bm{\ddots{u}} + \bm D \bm{\dot{u}} + \bm K \bm u = \bm f_{ext} \]
\item Rayleigh damping
\[ \bm D = \alpha \bm M + \beta \bm K \]
where $\alpha$ and $\beta$ are phenomenological parameters.
\end{itemize}

2. Computation of the mass matrix

Recall that $\bm M$ is defined
\[ \bm M = \sum_{e=1}^{n_{el}} \sum_{A=1}^{n_{en}} sum_{B=1}^{n_{en}} \bm L_{e,A}^t \int N^A \varphi N^B dV \bm L_{e,B} \]
\begin{equation}
\bm M_{AB}^e = \int N^A \varphi N^B dV
\label{eq:MassMatEntriesExact}
\end{equation}
We now use numerical integration: Gauss quad.
\begin{equation}
\bm M_{AB}^e = \sum_{q=1}^{n_{qp}} N^A \varphi N^B \bm I \text{det}(\bm J^e) w_q
\label{eq:MassMatEntriesNumerical}
\end{equation}

%TODO picture
\begin{tabular}{c || c | c | c}
Node & N & $\partial N / \partial \xi$ & $\partial N / \partial \eta$ \\
\hline
1 & $\frac{1}{4} (1-\xi)(1-\eta)$ & $-\frac{1}{4}(1-\eta)$ & $-\frac{1}{4}(1-\xi)$ \\
2 & $\frac{1}{4} (1+\xi)(1-\eta)$ & $ \frac{1}{4}(1-\eta)$ & $-\frac{1}{4}(1+\xi)$ \\
3 & $\frac{1}{4} (1+\xi)(1+\eta)$ & $ \frac{1}{4}(1+\eta)$ & $ \frac{1}{4}(1+\xi)$ \\
4 & $\frac{1}{4} (1-\xi)(1+\eta)$ & $-\frac{1}{4}(1+\eta)$ & $ \frac{1}{4}(1-\xi)$
\end{tabular}

%TODO picture

Connectivity
\begin{tabular}{c | c | c | c | c}
e/N& 1 & 2 & 3 & 4 \\
1 & 1 & 2 & 4 & 3 \\
2 & 5 & 3 & 4 & 6
\end{tabular}

Coordinates
\begin{tabular}{c | c | c}
N/t & $x_1$ & $x_2$ \\
1 & $1$ & $0$ \\
2 & $2$ & $0$ \\
3 & $\sqrt{2}/2$ & $\sqrt{2}/2$ \\
4 & $2$ & $2$ \\
5 & $0$ & $1$ \\
6 & $0$ & $2$
\end{tabular}

Compute $\bm M_{24}^1$

1. We need the coordinates of elements
\[ \bm X^e = 
\begin{pmatrix}
1 & 2 & 2 & 0.7071 \\
0 & 0 & 2 & 0.7071
\end{pmatrix} \] %TODO
2. Loop on quadrature points ($\rightarrow$ here we only compute one) \\
From equation \ref{eq:MassMatEntriesNumerical}:
\[ \bm M_{24}^1 = N^2 \varphi N^4 \bm I \text{det}(\bm J^e) w_1 \]
Coordinates of first quadrature ()%TODO
\[ N = \begin{Bmatrix}
0. \\ 0.1
\end{Bmatrix} \]
\[ \lambda^e = \begin{Bmatrix}
-0.3943 & -0.3943 \\
 0.3943 & -0.1057 \\
 0.1057 &  0.1057 \\
-0.1057 &  0.3943
\end{Bmatrix} \]

\[ \bm J^e = \bm X^e \lambda^e = \begin{bmatrix}
0.5309 & -0.1155 \\
0.1266 &  0.4902
\end{bmatrix} \]

\[ \text{det}(\bm J^e) = 0.276 \]
\[ dv^{e, I} = \text{det}(\bm J^e) w_q = 0.276 \]
\[ \bm M_{24}^{1, I} = N^2 \varphi N^4 dv^{1,1} \bm I \cong 0.0077 \bm I \]
\[ \bm M_{24}^{1, I} = \begin{bmatrix}
0.0077 & 0 \\
0 & 0.0077
\end{bmatrix} \]

ASSEMBLY OF MASS MATRIX
\begin{itemize}
\item[Option 1:]
\[ \bm M = \sum_{e=1}^{n_{el}} \sum_{A=1}^{n_{en}} \sum_{B=1}^{n_{en}} \bm L_{e,A}^t \sum_{q=1}^{n_{qp}} \bm M_{AB}^e \bm L_{e,B} \]
\item[Option 2:] Compute first a complete local mass matrix of the element $\bm M^e$ and then assemble into the global $\bm M$
\[ \bm M = \sum_{e=1}^{n_{el}} \bm M^e \]
\end{itemize}

3. The Central Difference Method - CDM

Recall from eq.1 %TODO noch nicht vorhanden. Spaeter linken
\[ \bm M \bm{\ddot{u}} + \bm K \bm u = \bm f_{ext} \]
%TODO picture
\begin{equation}
\bm{\dot{u}_n} = \frac{1}{2 \Delta t} [ \bm u_{n+1} - \bm u_{n-1} ]
\label{eq:FirstDerivNumerical}
\end{equation}
Taylor expansion
\begin{equation}
\bm u_{n+1} = \bm u_n + \Delta \dot{\bm u}_n + \frac{\Delta t^2}{2} \ddot{bm u}_n
\label{TaylorForwards}
\end{equation}
\begin{equation}
\bm u_{n-1} = \bm u_n - \Delta \dot{\bm u}_n + \frac{\Delta t^2}{2} \ddot{bm u}_n
\label{TaylorBackwards}
\end{equation}
From equation \ref{TaylorForwards} and \ref{TaylorBackwards}
\begin{equation}
\bm{\ddot{u}}_n = \frac{1}{\Delta t^2} [\bm u_{n-1} - 2 \bm u_n + \bm u_{n+1}]
\label{SecondDerivNumerical}
\end{equation}
Replace \ref{SecondDerivNumerical} into ???
\[ \bm M \frac{1}{\Delta t^2} [ \bm u_{n-1} - 2 \bm u_n + \bm u_{n+1}] + \bm K \bm u_n = \bm f_n^{ext} \]
\begin{equation}
\bm u_{n+1} = \Delta t^2 \bm M^{-1} [ \bm t_n^{ext} - \bm K \bm u_n ] + 2 \bm u_n - \bm u_{n-1}
= \bm u_n + \Delta t \left[ \frac{\bm u_n - \bm u_{n-1}}{\Delta t} + \Delta t^2 \bm M^{-1} \bm f_n
TODO \right]
\label{keineAhnungWas1}
\end{equation}
\begin{equation}
\bm v_n = \frac{\bm u_n - \bm u_{n-1}}{\Delta t} \neq \bm{\dot{u}}_n \text{on eq 4???TODO}
TODO
\label{keineAhnungWas2}
\end{equation}
\[ \bm u_{n+1} = \bm u_n + \Delta t \bm v_n + \Delta t^2 \bm{\ddot{u}}_n
= \bm u_n + \Delta t [ \bm v_n + \Delta t \bm{\ddot{u}}_n ] \]
\begin{equation}
\bm \hat{v}_n = \bm v_n + \Delta t \bm{\ddot{u}}_n
TODO
\label{keineAhnungWas3}
\end{equation}
Finally the update for the displacement is
\begin{equation}
\bm u_{n+1} = \bm u_n + \Delta t \bm{\hat{v}}_n
TODO
\label{keineAhnungWas4}
\end{equation}

\textbf{Remark:} \\
The update of $\bm u_{n+1}$ is \textit{explicit} because the predictions for $\bm u_{n+1}$ are dependent on information from previous time steps $\rightarrow ~ t_n,~t_{n-1}$

4. Algorithm for CDM
\begin{itemize}
\item Given $\bm u_0$, $\bm v_0$, $\bm f_0^{ext}$, $\Delta t$
\item Loop on time $t_0$ to $t_{final}$ \\
if $t_n = t_0$
\[ \bm u_{n-1} = \frac{\Delta t^2}{2} \bm M^{-1} \bm f_n^{ext} - \Delta t \bm{\dot{u}}_n \]
\[ \bm v_n = \frac{\bm u_n - \bm u_{n-1}}{\Delta t} \]
else
\[ \bm v_n = \bm{\hat{v}}_{n-1} \text{from eq 2 and 10 TODO} \]
end if
\[ \bm{\ddot{u}}_n = \bm M^{-1} \bm f_n \]
\[ \bm{\hat v} = \bm v_n + \Delta t \ddot{\bm u}_n \]
\[ \bm u_{n+1} = \bm u_n + \Delta t \bm{\hat v}_n \]
(We use $\bm{\hat v}_n$ in the following time step $\rightarrow$ save it! \\
End loop on time
\end{itemize}


\end{document}








