\documentclass[10pt,a4paper]{article}
\usepackage[utf8]{inputenc}
\usepackage{amsmath}
\usepackage{amsfonts}
\usepackage{amssymb}
\usepackage{graphicx}
\usepackage{bm}
\usepackage{units}
\usepackage{enumerate}

\begin{document}

\section{Einordnung der Strömungsmechanik/ Fluidmechanik in die Ingenieurswissenschaft}

%TODO TOADD Mindmap

\section{Anwendungsgebiete der Fluidmechanik}

\begin{itemize}
\item Bauwesen
\item Maschinenbau
\item Verfahrenstechnik
\item Verkehrswesen
\item Natur
\item \dots
\end{itemize}

\section{Eigenschaften von Fluiden}

\subsection{Molekularer Aufbau}

Materie besteht aus Atomen/ Molekülen vom Durchmesser $\sim 10^{-10} \text{m}$ \\
\begin{itemize}
\item[I)] geringer Abstand zwischen den Molekülen (hohe Dichte) \\
$\Rightarrow$ intermolekulare Kräfte beeinflussen das Stoffverhalten
\item[II)] großer Abstand (geringe Dichte) \\
$\Rightarrow$ intermolekulare Kräfte sind klein \\
$\Rightarrow$ Moleküle führen eine statische Bewegung infolge ihrer Thermischen Energie aus (kinetische Gastheorie)
\end{itemize}
Das unterschiedlich gewichtete Zusammenspiel dieser Bereiche führt zu den Aggregatzuständen (fest, flüssig, gasförmig).

\subsection{Definition und Kennzeichen}

Fluide sind Stoffe im flüssigen oder gasförmigen Aggregatszustand (Flüssigkeiten, Gase, Dämpfe). \\
\textbf{Gemeinsame Kennzeichen:}
\begin{itemize}
\item Moleküle sind leicht verschiebbar.
\item Eine andauernde Belastung führt zu einer andauernden Gestaltänderung.
\item Die an einem Fluid angreifende Schubspannung ist eine Funktion der Deformations\textit{geschwindigkeit}.
\end{itemize}
\textbf{Kennzeichen der Flüssigkeiten:}
\begin{itemize}
\item Zusammenhalt ohne bestimmte Form
\item Das Volumen einer Flüssigkeitsmenge ist praktisch konstant (dichtebeständig).
\item Ein gegebener Raum wird \textit{nicht} vollständig ausgefüllt.
\item Es bilden sich freie Oberflächen zu Gasen.
\item Begrenzende Wände werden benetzt.
%TODO add picture
\item Ohne Begrenzungsflächen bilden sich Tropfen.
\end{itemize}
\textbf{Kennzeichen von Gasen:}
\begin{itemize}
\item Der Molekülabstand ist größer als bei Flüssigkeiten.
\item kein Zusammenhalt
\item Das Volumen einer Gasmenge ist vom Zustand abhängig ($p$, $T$) (Druck, Temperatur).
\item Begrenzende Wände werden benetzt. \\
$\Rightarrow$ "Haften an Wänden"
\end{itemize}
\textbf{Kennzeichen eines Fluids als Kontinuum:}
\begin{itemize}
\item Ein Fluid ist auch im kleinsten Volumenelement $dV$ homogen verteilt.
\item Der mittlere Molekülabstand sowie die mittlere freie Wegelänge der Moleküle sind klein gegenüber den Abmessungen (z.B. $dV$) $\widehat{=}$ Knudsen-Zahl $Kn = \frac{\lambda}{l}$ ($\lambda$: mittlere freie Wegelänge (z.B. zwischen 2 Stößen); $l$: Abmessungen des zu berechnenden Körpers) \\
Bedingung für Kontinuum: $Kn < 0.01$
\end{itemize}

\subsection{Zustandsgrößen}

\textbf{Thermische Zustandgrößen}
\begin{itemize}
\item Mase $m$, $[m] = \text{kg}$
\item Volumen $V$, $[V] = \text{m}^3$
\item Temperatur $T$, $[T] = \text{K}$ \\
\phantom{Temperatur} $\vartheta$, $[\vartheta] = {^\circ} \text{C}$
\item Druck $p$, $[p] = \text{Pa}$
\end{itemize}
\textbf{Abgeleitete kalorische Zustandsgrößen}
\begin{itemize}
\item innere Energie $U$, $[U] = \text{J}$
\item Enthalpie $H$, $[H] = \text{J}$
\item Entropie $S$, $[S] = \frac{\text{J}}{\text{K}}$
\end{itemize}
\textbf{Nomenklatur}
\begin{itemize}
\item Intensive Zustandsgrößen sind unabhängig von der Stoffmenge; z.B. $p$, $T$
\item Extensive Zustandsgrößen sind abhängig von der Stoffmenge; z.B. $H$, $S$, $U$
\item spezifische Zustandsgrößen $\widehat{=} \frac{\text{extensive Größe}}{Masse}$; i.d.R. kleine Buchstaben; z.B. $h$, $s$, $u$
\item molare Zustandsgrößen $\widehat{=} \frac{\text{extensive Größe}}{Molanzahl}$
\item -"dichte" $\widehat{=} \frac{\text{extensive Größe}}{Volumen}$; z.B. $\text{Dichte} = \frac{\text{Masse}}{\text{Volumen}} = \rho$, $[\rho] = \frac{\text{kg}}{\text{m}^3}$
\end{itemize}

\subsubsection{Masse und Volumen}

\begin{itemize}
\item Masse $m$, $[m] = \text{kg}$
\item Teilchenanzahl $N$, $[N] = 1$
\item Stoffmenge $\nu = \frac{N}{N_A}$, $N_A$: Avogadro-Konstante $N_A = 6,022 \cdot 10^{23} \text{mol}^{-1}$
\end{itemize}
\textbf{Bezogene extensive Zustandsgrößen $V$ und $m$}
\begin{itemize}
\item Dichte: $\lim_{\Delta V \rightarrow 0} \frac{\Delta m}{\Delta V} = \frac{dm}{dV} = \rho$, $[\rho] = \frac{\text{kg}}{\text{m}^3}$
\item spezifisches Volumen: $v = \frac{1}{\rho}$, $[v] = \frac{\text{m}^3}{\text{kg}}$
\end{itemize}

\subsubsection{Temperatur}

\begin{itemize}
\item intensive Zustandsgröße
\item Maß für die kinetische Energie der Teilchen $E_{kin} = \frac{f}{2} k T = \frac{1}{2} m_A \bar{c}^2$ \\
$f$: Anzahl der Freiheitsgerade (z.B. $f=3$ für einatomige Gase) \\
$k = 1,38 \cdot 10^{-23} \frac{\text{J}}{\text{K}}$ Boltzmann-Konstante
\item Anmerkungen: $\vartheta$: Temperatur in Grad Celsius \\
\phantom{Anmerkungen:} $T = \theta + 273,15$
\end{itemize}

\subsubsection{Druck}

\begin{itemize}
\item intensive Zustandsgröße
\item Der Druck ist der Quotient aus der Kraft $F$ und der Fläche $A$, auf der die Kraft senkrecht steht ($\perp$) \\
$p = \frac{F}{A}$, $[p] = \frac{\text{N}}{\text{m}^2} = \frac{\text{kg} \text{m}}{\text{s}^2} = \text{Pa}$, alt: $1 \text{bar} = 10^5 \text{Pa}$ \\
$1 \text{mm} \text{W} \text{S} = 9,81 \text{Pa}$ ????? %TODO was ist mmWS?
\item Druck ist eine skalare Größe.
\end{itemize}

\subsubsection{Innere Energie}

\[ U := E - E_{kin} - E_{pot} \]
\begin{itemize}
\item[$U$] Innere Energie $\widehat{=}$ Summe der Energien aller Moleküle (Translation infolge $T$, Rotation, Schwingungen) \\
Extensive Zustandsgröße
\item[$E$] Gesamtenergie einer fluiden Phase
\item[$E_{kin}$] kinetische Energie, die zur Position des Systems als "Ganzes" gehört
\item[$E_{pot}$] potentielle Energie, die zur Bewegung des Systems als "Ganzes" gehört
\end{itemize}
\begin{itemize}
\item[$u$] $:= \frac{U}{m}$ spezifische innere Energie
\item[$c_V$] $:= \big( \frac{\partial u}{\partial T} \big)_V$ spezifische isochore Wärmekapazität
\item[$u(T, p)$] kalorische Zustandsgleichung
\end{itemize}

\subsubsection{Enthalpie}

\[ H = U + p V \]
\begin{itemize}
\item[$H$] Enthalpie, extensive Zustandsgröße
\item[$h$] $:= \frac{H}{m} = u + p v$ spezifische Enthalpie
\item[$pV$] Volumenänderungsarbeit, die erforderlich ist, das Volumen $V$ in einem Raum mit dem Druck $p$ zu fördern
\item[$c_p$] $:= \big( \frac{\partial h}{\partial T} \big)_p$ spezifische isobare Wärmekapazität
\item[$h(p, T)$] kalorische Zustandsgleichung
\end{itemize}
Hinweis:
%TODO add picture
$\dot{m} (h + \frac{c^2}{2} + gz)$ $\widehat{=}$ der einem Kontrollvolumen mit dem Massenstrom $\dot{m}$ zugeführten Leistung

%TODO
%-- Einschub --
%Quellen:
%
%Schroeder: Thermal Physics
%
%Sigloch: Technische Fluidmechanik
%
%Herwig: Strömungsmechanik
%
%--------

%Jetzt kommt 3.3.6

\subsubsection{Entropie}

$S$ externe Zustandsgröße \\
$s = \frac{S}{m}$ spezifische Entropie \\
Beispiel: Einstein-Solids \\
\begin{itemize}
\item[$q$] Energiequanten
\item[$N$] Energie"speicher"
\item[$\Omega$] Multiplicity $\widehat{=}$ Anzahl der Möglichkeiten die Quanten in den Speicher zu verteilen \\
$\Omega (N, q) = {q + N - 1 \choose q} = \frac{(q + N - 1)!}{q! (N - 1)!}$
\end{itemize}

Als Beispiel seien $N_A = 3$, $N_B = 3$, $q_t = q_A + q_B = 4$ gewählt. Dann gibt es folgene Möglichkeiten: \\
\begin{tabular}{c c | c c | c c}
\multicolumn{2}{c |}{$N_A$} & \multicolumn{2}{c |}{$N_B$} & & \\
$q_A$ & $\Omega_A$ & $q_B$ & $\Omega_B$ & $q_t$ & $\Omega_A \Omega_B$ \\ \hline
0 &  1 & 4 & 15 & 4 & 15 \\
1 &  3 & 3 & 10 & 4 & 30 \\
2 &  6 & 2 &  6 & 4 & 36 \\
3 & 10 & 1 &  3 & 4 & 30 \\
4 & 15 & 0 &  1 & 4 & 15
\end{tabular}

Angenommen es sind $1$ Energiequant in $A$ und $3$ in $B$ gespeichert. Das ist der Macrostate. Ein von 3 möglichen Microstates in $A$ ist dann die Verteilung $0,1,0$ und in $B$ ist dann $0,2,1$ einer von zehn moeglichen Microstates. Unabhängig von dieser inneren Verteilung ist die Multiplicity immer $\Omega = \Omega_A \Omega_B = 30$ \\
%z.B.: \\
%Macrostate: $q_A = 1$, $q_B = 3$ \\
%Microstate: $010$ $021$ (z.B.) \\
%Multiplicity: $\Omega = \Omega_A \Omega_B = 30$

\textbf{2. Hauptsatz der Thermodynamik}

Jedes große System im Gleichgewicht wird einen "Macrostate" mit der höchsten "Multiplicity" einnehmen.

$S := k \ln(\Omega)$ ($k$ Boltzmann Konstante)

\textbf{Hinweise:}
\begin{itemize}
\item $dS = \frac{dQ}{T}$ Zufuhr von Energiequanten erhöht die Multiplicity $\Omega$
\item Für jedes adiabate System gilt $dS \geq 0$
\item Gibb'sche Gleichung $ds = \frac{du + pdv}{T} = \frac{dh - vdp}{T} = \frac{dq}{T}$
\end{itemize}

\subsection{Thermische Zustandsgleichungen}

Zustandsgleichungen stellen eien stoffspezifischen Zusammenhang zwischen den Zustandsgrößen dar.
\begin{itemize}
\item thermische Zustandsgleichung (z.B. $f(p, V, T)$)
\item kalorische Zustandsgleichung (z.B. $f(p, V, T, H, S, U$)
\end{itemize}
Viele Fluide besitzen eigene Zustandsgleichungen!

\subsubsection{Gibb'sche Phasenregel}

\[F = K + 2 - P\]
\begin{itemize}
\item[$F$] Freiheitsgerade (unabhängige Zustandsgrößen)
\item[$K$] Anzahl der Komponenten
\item[$P$] Anzahl der Phasen (fest, flüssig, gasförmig)
\end{itemize}
z.B. flüssiges Wasser und Wasserdampf \\
$\left.
\begin{array}{ll}
K = 1 \\ 
P = 2
\end{array}
\right\}
\Rightarrow F = 1 \rightarrow P_D = f(T) ~~ \text{Dampfdruckkurve}
$

\subsubsection{Ideale Gasgleichung}

\textbf{ideale Gase:}
\begin{itemize}
\item keine intermolekularen Kräfte (van der Waals-Kräfte)
\item Moleküle haben kein Volumen
\end{itemize}

\[p V = \nu Q T\]

\begin{itemize}
\item[$\nu$] Stoffmenge
\item[$Q$] $= 8,314 \frac{\text{J}}{\text{mol}~\text{K}}$ ~Universelle Gaskonstante
\end{itemize}

\[p V = m R T\]
\begin{itemize}
\item[$R$] spezifische Gaskonsante
\end{itemize}

\[p = \rho R T\]
\begin{itemize}
\item[$R$] $= \frac{Q}{M}$, $M$ Molmasse
\end{itemize}

\subsubsection{Reale Gase}

Erweiterung der idealen Gasgleichung
\begin{itemize}
\item intermolekulare Kräfte, indem der reale Druck um den Binnendruck vergrößert wird \\
$p_{real} < r_{ideal}$
\item reales Gasvolumen $V$ wird um das Kovolumen verkleinert \\
$V_{real} > V_{ideal}$
\end{itemize}
Man erhält die van der Waals'sche Zustandsgleichung:
\[
\underbrace{\Big( p + \overbrace{\frac{a}{V_A^2}}^{\substack{Binnen-\\druck}} \Big)}_{\substack{P_{ideal}}} \underbrace{\big( V_A - \overbrace{b}^{\substack{Ko-\\volumen}} \big)}_{\substack{V_{ideal}}} = QT
\]

$a$ und $b$ sind van der Waals-Konstanten (s. Kuchling: Taschenbuch der Physik) VDI Wärmeatlas ??????

\textbf{Alternativen:}
\begin{itemize}
\item erweiterte Gasgleichung \\
\[p V = m R T Z(p, T)\] \\
$Z$ Realgaskoeffizient (z.B. $Z(p, T)$)
\item Virialgleichung \\
\[\frac{p V_A}{Q T} = 1 + \frac{a_2}{V_A} + \frac{a_3}{V_A^2} + \dots\] \\
$a_2$, $a_3$ Virialkoeffizienten
\end{itemize}

\subsection{Transporteigenschaften}

z.B. Diffusion (Fick'sches Gesetz: Stoff-/ Massentransport), Reibung (Impulstransport), Wärme (Fouriesches Gesetz: Energietransport)

\subsubsection{Viskosität (früher Zähigkeit) Newtonscher Fluide}

\textbf{Definition:} Die Viskosität ist die Eigenschaft eines Fluides, beim Verformen eine Spannung aufzunehmen, die von der Deformationsgeschwindigkeit abhängt. \\
\textbf{Frage:} Welcher Zusammenhang besteht zwischen der Schubspannung im Fluid und der Deformationsgeschwindigkeit? \\
\textbf{Versuch:} Couette-Versuch für laminare Strömung ($Re_{krit} = 1300 > Re$) \\
(Schleppen einer Platte mit der Geschwindigkeit $C_x$)

%TODO ADD picture
\begin{itemize}
\item[$F$] Schleppkfraft
\item[$A$]
\item[$C_x$]
\item[$c_x(y)$]
\item[$h$]
\end{itemize}
RB's - Schichtenströmung ohne Druckgradienten \\
$\frac{\partial p}{\partial x} = \frac{\partial p}{\partial y} = 0$ (1D Strömung) für $Re < Re_{krit} = \frac{h C_x \rho}{\eta}$ \\
Versuchsergebnisse:
\begin{itemize}
\item lineare Geschwindigkeitsverteilung $c_x(y)$
\item $F \sim C_x$
\item $F \sim A$
\item $F \sim h^{-1}$
\end{itemize}

%TODO geschweifte Klammern und so

$F \sim \frac{C_x A}{h}$ \\
$F = \eta \frac{C_x A}{h}$, $\eta$ Proportionalitätsfaktor \\
Mit $\tau = \frac{F}{A} = \eta \frac{C_x}{h} = \eta \frac{d c_x}{d y}$

%TODO TOADD

\textbf{Anmerkungen:}
\begin{itemize}
\item Die Schubspannung $\tau$ wird in jedem Schritt $y = \text{konst}$ übertragen.
\item Die dynamische Viskosität $\eta$ ist eine Stoffeigenschaft die primär temperaturabhängig und kaum druckabhängig ist. \\
$\eta = f(T)$ Gase: Sutherland Formel, Flüssigkeiten: empirische Formeln
%TODO TOADD Graph
\item Die kinematische Viskosität $\nu$ ist definiert als
\[\nu := \frac{\eta}{\rho}\]
$[\nu] = \frac{m^2}{s}$
\item Für turbulente Strömungen gilt
\[ \tau = (\eta + \eta_T) \frac{dc}{dy} = \eta \frac{dc}{dy} - \rho \overline{c_x' c_y'} \]
= laminarer Anteil + turbulente Scheinreibung \\ %TODO TOADD geschweifte Klammern
Ansatz: $c_x = \overline{c_x} + c_x'$, $c_y = \overline{c_y} + c_y'$, $c_z = \overline{c_z} + c_z'$
%TODO TOADD Bilder/ Graphen
\end{itemize}

\subsubsection{Viskosität der Nicht-Newton'scher Fluide}

%TODO TOADD Graph
\begin{itemize}
\item Newton'sches Fluids \\ Luft, Waser, \dots
\item plastisches Fluid (Bingham-Fluid) \\ Zahnpasta, Honig
\item pseudoplastisches Fluid \\ Schmelzen, Hochpolymere
\item dilatantes Fluid \\ Klebstoffe, nasser Sand
\item ideales Fluid ($\eta = 0$)
\end{itemize}

\section{Statik der Fluide}

Statik bedeutet, dass das Fluid sich in Ruhe befindet: \\
Strömungsgeschwindigkeit: $\overline{c} = (c_x c_y c_z) = 0 ~~ \Rightarrow \frac{\partial c}{\partial x} = \frac{\partial c}{\partial y} = \frac{\partial c}{\partial z} = 0$ \\
$\Rightarrow$ keine Schubspannung im Fluid: $\tau \sim \frac{\partial c}{\partial y_t ??????} = 0$

\subsection{Grundgleichung der Fluidstatik}

\textbf{Annahmen:}
\begin{itemize}
\item ruhendes kompressibles Fluid ($\rho$ veränderlich)
\item Zustandsgrößen sind von z abheangig (z.B. $\rho (z)$) \\
%TODO TOADD Bild
\end{itemize}

Kräftegleichgewicht in z-Richtung \\
$\sum F_z = 0 = p dA - (p + dp) dA - \rho(z) dA dz g(z)$ (letzter Term ist $dG$) %TODO geschweifte Klammer
\[ \frac{dp}{dz} = -\rho(z) g(z) \]
Grundgleichung der Fluidstatik

\subsection{Hydrostatik}
\subsubsection{Grundgleichung der Hydrostatik}
\textbf{Annahmen:}
\begin{itemize}
\item ruhendes Fluid für $\rho = \text{konst}$ $\widehat{=}$ inkommpressibles Fluid
\item konstante Erdbeschleunigung $g$
\end{itemize}
%TODO TOADD Bild

$\frac{dp}{dz} = - \rho g$ \\
$\int_p^{p_0} dp = - \int_0^h \rho g dz$ \\
$p_0 - p = - \rho g h$ \\
\[ p = p_0 + \rho g h \]
\[ p(z) = p_0 + \rho g z' \]
Grundgleichung der Hydrostatik

\textbf{Anmerkungen:}
\begin{itemize}
\item In Punkten gleicher Tiefe herscht der gleiche Druck.
\item Druck wächst proportional zur Tiefe (Proportionalitätsfaktor $\widehat{=}$ Wichte $\gamma = \rho g$)
\end{itemize}

\end{document}















